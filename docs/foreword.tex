\thispagestyle{plain}

\section{Förord}
Detta projekt har utvecklats som en del av kursen DSD400, med fokus på distribuerade system, databaser och blockkedjor.

Projekt: YouTube Transcription Service
Projektperiod: 8 februari - 17 mars

Arbetsfördelning:
Hampus Avekvist, Andreas Berger

Gemensamma Ansvarsområden (Andreas Berger & Hampus Avekvist):
Beskrivning och design av systemet (8 februari - 10 mars).
Upprättande av användningsfall (8 februari).
Implementation, inklusive transkriberingsmotor och hantering av transkriberingsförfrågningar (20 februari - 17 mars).
Utarbetande av projektets presentation och rapport, från inledning till slutsatser (14 februari - 17 mars).

Hampus Avekvist:
Backend Utveckling (20 februari - 17 mars):
    Skapande av applikations- och API-routing.
    Implementering av transkriberingsförfrågan, inloggnings- och registreringsfunktionaliteter.
E-posttjänst (9 mars - 17 mars):
    Utveckling och integration med SMTP-tjänsten för att skicka e-post via Postfix.
Köhantering (9 mars - 17 mars):
    Uppsättning av e-post- och transkriberingsköer.
Databashantering (15 mars):
    Hantering och upplägg av transkriberingar i databasen.

Andreas Berger:
Frontend Utveckling (20 februari - 15 mars):
    Implementering av transkriberingsförfrågan.
    Utveckling av registrerings- och inloggningsfunktionaliteter.
    Anpassning av formulärets bredd till bildens storlek (#28, 27 februari).
Databas (6 mars - 17 mars):
    Skapande av användar- och transkriberingstabeller.