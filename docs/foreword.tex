\thispagestyle{plain}

\section{Förord}
Detta projekt har utvecklats som en del av kursen DSD400, med fokus på
distribuerade system, databaser och blockkedjor. Tack Fredrik, Felicia, Julia
och Stina för att ni har kunnat läsa igenom och kommentera på rapporten.

Projektperiod: 8 februari - 17 mars

Gruppmedlemmar:
\begin{itemize}
    \item Hampus Avekvist
    \item Andreas Berger
\end{itemize}

Gruppgemensamma ansvarsområden:
\begin{itemize}
    \item Beskrivning och design av systemet
    \item Upprättande av användningsfall
    \item Implementation, inklusive transkriberingsmotor och hantering av transkriberingsförfrågningar
    \item Utarbetande av projektets presentation och rapport, från inledning till slutsatser
\end{itemize}

Hampus Avekvist:
\begin{itemize}
    \item Backendutveckling
    \item Skapande av applikations- och API-routing
    \item Implementering av transkriberingsförfrågan
    \item Köhantering
    \item Uppsättning av transkriberingskö
    \item Databashantering
    \item Hantering och upplägg av transkriberingar i databasen
\end{itemize}

Andreas Berger:
\begin{itemize}
    \item Frontendutveckling
    \item Implementering av transkriberingsförfrågan
    \item Databas
    \item Skapande av transkriberingstabell
\end{itemize}
