\thispagestyle{plain}

\section{Förord}
Detta projekt har utvecklats som en del av kursen DSD400, med fokus på distribuerade system, databaser och blockkedjor.

Projektperiod: 8 februari - 17 mars

Gruppmedlemmar:
\begin{itemize}
    \item Hampus Avekvist
    \item Andreas Berger
\end{itemize}

Gruppgemensamma ansvarsområden:
\begin{itemize}
    \item Beskrivning och design av systemet (8 februari - 10 mars).
    \item Upprättande av användningsfall (8 februari).
    \item Implementation, inklusive transkriberingsmotor och hantering av transkriberingsförfrågningar (20 februari - 17 mars).
    \item Utarbetande av projektets presentation och rapport, från inledning till slutsatser (14 februari - 17 mars).
\end{itemize}

Hampus Avekvist:
\begin{itemize}
    \item Backendutveckling (20 februari - 17 mars):
    \item Skapande av applikations- och API-routing.
    \item Implementering av transkriberingsförfrågan, inloggnings- och registreringsfunktionaliteter.
    \item E-posttjänst (9 mars - 17 mars):
    \item Utveckling och integration med SMTP-tjänsten för att skicka e-post via Postfix.
    \item Köhantering (9 mars - 17 mars):
    \item Uppsättning av e-post- och transkriberingsköer.
    \item Databashantering (15 mars):
    \item Hantering och upplägg av transkriberingar i databasen.
\end{itemize}

Andreas Berger:
\begin{itemize}
    \item Frontendutveckling (20 februari - 15 mars):
    \item Implementering av transkriberingsförfrågan.
    \item Utveckling av registrerings- och inloggningsfunktionaliteter.
    \item Databas (6 mars - 17 mars):
    \item Skapande av användar- och transkriberingstabeller.
\end{itemize}
