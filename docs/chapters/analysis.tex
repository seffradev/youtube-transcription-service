\chapter{Analys}
Potential: Projektet illustrerar den stora potentialen i att använda AI för att öka tillgängligheten och användbarheten av videomaterial på plattformar som YouTube. Genom att tillhandahålla en automatiserad tjänst för transkribering av videor, adresserar det direkt behovet av textversioner av videoinnehåll. Detta gagnar inte bara personer med hörselnedsättning utan även akademiker, studenter, och andra användargrupper som kan ha nytta av texttranskriptioner.

Utmaningar: En av de främsta utmaningarna som projektet står inför är dess förmåga att tolka och återge kroppsspråk och icke-verbala signaler. Dessa aspekter av mänsklig kommunikation, som inte kan fångas genom texttranskriptioner ensamma, understryker nuvarande begränsningar i AI-teknologins förmåga att fullständigt förstå och återge mänsklig interaktion.

Tekniska Aspekter: Analysen framhåller de tekniska komponenterna i systemet, som användningen av Apache Kafka för hantering av köer av transkriptionsförfrågningar och Apache ZooKeeper för koordinering av konfigurationsdata. Dessa element visar på en robust systemarkitektur som är designad för att hantera och processa stora mängder data effektivt.

Framtidspotential och Vidareutveckling: Analysen pekar också på framtida möjligheter för att utveckla och förbättra systemet. Detta inkluderar utforskandet av metoder för att integrera tolkningar av kroppsspråk i transkriptionerna och att utöka tjänsten för att stödja fler format och funktioner, som segmenterad transkribering.
