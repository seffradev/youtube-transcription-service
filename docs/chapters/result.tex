\chapter{Resultat}

\section{Systemdesign}

\subsection{Allmän systembeskrivning}
Systemet ska vara en webbapplikation som tillåter användare att transkribera
YouTube-videor. Användare ska kunna skapa ett konto och logga in för att
transkribera videor. Användare ska kunna skapa nya transkriberingar genom att
mata in en URL till en YouTube-video. Andra användare ska kunna hämta samma
transkribering genom att mata in samma URL. 

\subsection{Arkitektur}

\subsubsection{Komponenter}

\paragraph{Transkriberingskö}
Transkriberingskön innehåller meddelanden med URL:er till YouTube-videor som
ska transkriberas. I dessa meddelanden ingår även information om vem som
begärde transkriberingen. Transkriberingskön är en separat komponent som drivs
av Apache Kafka.

\paragraph{E-postkö}
E-postkön innehåller meddelanden med e-postadresser och transkriberingar som
ska skickas till dessa adresser. Används för att se till att alla användare
får de transkriberingar de begär om systemet är överbelastat eller kraschar.
E-postkön är en separat komponent som drivs av Apache Kafka.

\paragraph{Transkriberingsmotor}
Transkriberingsmotorn ska vara en separat komponent som hämtar från
transkriberingskön. Efter hämtat meddelande från kön ska motorn hämta videon
från YouTube och transkribera den. När transkriberingen är klar ska motorn
lägga till den i databasen och skicka en notis till e-postkön att
transkriberingen är klar. Sist ska motorn radera videon från disk för att
frigöra utrymme. Transkriberingsmotorn implementeras i programmeringsspråket
Python med hjälp av OpenAI:s Whisper.

\paragraph{E-posttjänst}

\paragraph{Backend}

\paragraph{Frontend}

\paragraph{Databas}

\subsubsection{Kommunikation}

\subsubsection{Säkerhet}

\subsection{Användningsfall}

\subsubsection{Skapa konto}

\subsubsection{Logga in}

\subsubsection{Skapa transkribering}

\subsubsection{Hämta transkribering}

\subsection{Tidsplan}
