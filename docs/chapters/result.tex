\chapter{Resultat}
\label{ch:result}

\section{Allmän systembeskrivning}
\label{sec:general-system-description}
Det färdiga systemet är en webbapplikation som tillåter användare att
transkribera YouTube-videor. Användare kan autentisera sig med ett hårdkodat
användarnamn och lösenord via HTTP Basic Authentication \cite{RFC7617}. Efter
autentisering får användaren utrymme att begära transkriberingar. Varje
transkribering skickas till en kö som hanteras av Apache Kafka
\cite{ApacheKafka}. En separat komponent, transkriberingsmotorn, hämtar
transkriberingsförfrågningar från kön och transkriberar videorna med hjälp av
OpenAI:s Whisper \cite{OpenAIWhisper}. När transkriberingen är klar finns den
tillgänglig för användaren att hämta via webbapplikationen.

\begin{figure}[h]
    \centering
    \includesvg[width=0.8\textwidth]{images/system-nodes.svg}
    \caption{De noder som ingår i systemet}
    \label{fig:system-nodes}
\end{figure}

\section{Komponenter}
\label{sec:components}

\subsection{Transkriberingskö}
\label{sec:transcription-queue}
Transkriberingskön innehåller meddelanden med URL:er till YouTube-videor som
ska transkriberas. I dessa meddelanden ingår även information om vem som
begärde transkriberingen. Transkriberingskön är en separat komponent som drivs
av Apache Kafka \cite{ApacheKafka}.

\subsection{Transkriberingsmotor}
\label{sec:transcription-engine}
Transkriberingsmotorn är en separat komponent implementerad i Python
\cite{PythonSoftwareFoundation2024} som hämtar meddelanden från
transkriberingskön. Varje meddelande innehåller ett ID som relaterar till en
YouTube-video. Transkriberingsmotorn hämtar videor från YouTube och
transkriberar dem med hjälp av OpenAI:s Whisper med inställning \verb|small|
\cite{OpenAIWhisper}. När transkriberingen är klar läggs den till i databasen
och videofilen raderas från disk. Flödesdiagram för hur tjänsten arbetar är
enligt Figur \ref{fig:transcription-flowchart}. 

\subsection{Webbapplikation}
\label{sec:web-application}
Webbapplikationen, implementerad i Go med ramverket Gin \cite{GinGonic},
tillåter klienter att autentisera sig och begära transkriberingar.
Webbapplikationen är en separat komponent som vid förfrågan kontrollerar om
klienten är autentiserad och om efterfrågad transkribering inte redan finns i
databasen. Om transkriberingen inte finns i databasen skickas ett meddelande
till transkriberingskön som transkriberingsmotorn lyssnar på. För
flödesdiagram hur webbapplikationen fungerar, se Figur
\ref{fig:system-flowchart}.

Tillhörande webbapplikationen finns ett grafiskt gränssnitt och är
implementerad i HTML, CSS och JavaScript. Det låter klienter autentisera sig
och begära transkriberingar. 

\subsection{Databas}
\label{sec:database}
Databasen är en separat nod som lagrar transkriberingar. Databasen är en
instans av den relationella databasen MySQL och driftas på Lightsail på Amazon
Web Services \cite{AmazonWebServices2024Lightsail}.

\section{Kommunikation}
\label{sec:communication}
Namnhantering för tjänsterna sköts via Apache ZooKeeper och
Docker Compose som har en inbyggd DNS-översättning \cite{ApacheZooKeeper,
DockerNetworkingDocumentation2024}. Tjänsterna kommunicerar med varandra via
HTTP och Kafka Topics. Webbapplikationen är en producent till
transkriberingskön, medan transkriberingsmotorn är en konsument. 

\section{Säkerhet}
\label{sec:security}
Systemet bygger på HTTP Basic Authentication \cite{RFC7617} för att autentisera
användare. Autentiseringen är hårdkodad till användarnamnet \verb|foo| och
lösenordet \verb|bar|. Kommunikationen mellan klienten och servern är
krypterad med HTTPS via Cloudflare.

\section{Skalbarhet}
\label{sec:scalability}
Systemet i Figur \ref{fig:system-nodes} består av flera separata komponenter
som kan köras på olika maskiner eller i olika Docker-containrar. Det är möjligt
att lägga till fler instanser av transkriberingsmotorn, webbapplikationen,
Apache Kafka, MariaDB och Apache ZooKeeper för att öka kapaciteten. Det är
också möjligt att lägga till en lastbalanserare och en proxy för att dela upp
trafiken till flera instanser av webbapplikationen. Systemet är byggt enligt
arkitekturtermen ''microservices'' \cite{MicroservicesIO2024} och ger
horisontell skalbarhet samt feltolerans.
