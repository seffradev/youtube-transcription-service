\chapter{Resultat}

\section{Systemdesign}

\subsection{Allmän systembeskrivning}
Systemet ska vara en webbapplikation som tillåter användare att transkribera
YouTube-videor. Användare ska kunna skapa ett konto och logga in för att
transkribera videor. Användare ska kunna skapa nya transkriberingar genom att
mata in en URL till en YouTube-video. Andra användare ska kunna hämta samma
transkribering genom att mata in samma URL.

\subsection{Arkitektur}

\begin{figure}[h]
    \centering
    \includesvg[width=0.8\textwidth]{images/system-nodes.svg}
    \caption{Noderna som ingår i systemet}
    \label{fig:system-nodes}
\end{figure}

\subsubsection{Komponenter}

\paragraph{Transkriberingskö}
Transkriberingskön innehåller meddelanden med URL:er till YouTube-videor som
ska transkriberas. I dessa meddelanden ingår även information om vem som
begärde transkriberingen. Transkriberingskön är en separat komponent som drivs
av Apache Kafka.

\paragraph{Transkriberingsmotor}
Transkriberingsmotorn ska vara en separat komponent som hämtar från
transkriberingskön. Efter hämtat meddelande från kön ska motorn hämta videon
från YouTube och transkribera den. När transkriberingen är klar ska motorn
lägga till den i databasen och skicka en notis till e-postkön att
transkriberingen är klar. Sist ska motorn radera videon från disk för att
frigöra utrymme. Transkriberingsmotorn implementeras i programmeringsspråket
Python med hjälp av OpenAI:s Whisper. För flödesdiagram, se Figur
\ref{fig:transcription-flowchart}.

\paragraph{Webbapplikation}
Webbapplikationen ska vara en separat komponent som tillåter användare att
skapa ett konto, logga in och begära transkriberingar via ett RESTful API.
Webbapplikationen ska även hantera autentisering och auktorisering av
användare. Webbapplikationen implementeras i programmeringsspråket Go med
hjälp av ramverket Gin. För flödesdiagram, se Figur \ref{fig:system-flowchart}.

\paragraph{Grafiskt gränssnitt}
Det grafiska gränssnittet ska vara en separat komponent som tillåter användare
att skapa ett konto, logga in och begära transkriberingar. Gränssnittet ska
även visa hur många polletter användaren har kvar och hur många polletter en
transkribering kommer att kosta. Det grafiska gränssnittet implementeras i
programmeringsspråket JavaScript, HTML och CSS utan något ramverk.

\paragraph{Databas}
Databasen ska vara en separat komponent som lagrar användare, transkriberingar
och polletter. Databasen är en instans av den relationella databasen MariaDB.

\subsubsection{Kommunikation}
Namnhantering för tjänsterna ligger via Apache ZooKeeper och Docker Compose
som har en inbyggd DNS-server. Tjänsterna kommunicerar med varandra via HTTP
och Kafka. Webbapplikationen är en producent av transkriberingskön, medan
transkriberingsmotorn är en konsument. Transkriberingsmotorn är en producent
av e-postkön, medan e-posttjänsten är en konsument. Alla konsumenter är så
kallade ''komplicerade konsumenter'' som behöver hantera vilket meddelande som
är nästa att processera själva.

\subsubsection{Säkerhet}
Systemet bygger på HTTP Basic Authentication för att autentisera användare.
Vid autentisering får användaren en enkel pollett som används för att
auktorisera användaren att begära transkriberingar. Polletten är inte en JSON
Web Token (JWT) utan en enkel sträng som lagras i en kaka. Polletten är enkel
att stjäla och användas av en obehörig användare. 

I databasen lagras lösenord inte i klartext utan i hashad form med hjälp av \verb|bcrypt|. 

\subsubsection{Skalbarhet}
Systemet som illustreras i Figur \ref{fig:system-nodes} är skalbart. Det går
att lägga till fler instanser av transkriberingsmotorn och e-posttjänsten för
att öka kapaciteten. Det går även att lägga till fler instanser av Apache
Kafka, MariaDB och Apache ZooKeeper samt en lastbalanserare och en proxy för
att öka webbapplikationens kapacitet. Byggs systemet enligt en sådan modell
används arkitekturtermen ''microservices'' och ger horisontell skalbarhet samt
ökad feltolerans. 
