\chapter{Introduktion}

\section{Bakgrund}
Med mängden användare av digitalt videomaterial på plattformar som YouTube har
behovet av tillgängliga textversioner blivit alltmer angeläget. Denna
utveckling reflekteras i akademiska studier som betonar vikten av
transkribering inom forskning, där det möjliggör en mer djupgående analys av
material och en förbättrad tillgänglighet för olika användargrupper
\cite{RevBlog2019}. Transkriberingens roll inom akademisk forskning
framhålls för att tillhandahålla en detaljerad och exakt representation av
muntliga data, vilket är avgörande för kvalitativ forskning
\cite{OxfordAcademicND}.

Trots tillgängligheten av olika transkriberingstjänster stöter många användare
på utmaningar relaterade till kostnad, noggrannhet och tillgänglighet.
Existerande lösningar kan kräva betydande tid för granskning och kan vara
budgetkrävande, vilket lägger hinder för forskare som arbetar under tidspress
\cite{RevBlog2019}. Projektet syftar till att tackla dessa utmaningar genom
att utveckla en kostnadseffektiv, noggrann och lättillgänglig
transkriberingstjänst för YouTube-videor. Effektiva transkriberingstjänster
kan spela en kritisk roll i att förbättra tillgången på forskningsdata och
stödja forskare i deras arbete, vilket ytterligare bekräftas av Happy Scribe
som framhäver transkriberingens värde i specialutbildningsforskning och hur
det kan förbättra förståelsen för komplexa termer och terminologi
\cite{HappyScribeND}.

\subsection{Nomenklatur}
För att underlätta förståelsen av projektet och dess tekniska aspekter
presenteras här en kort förklaring av de termer som används i rapporten:
\begin{itemize}
    \item Hypertext Transfer Protocol (HTTP): Ett protokoll för överföring av
    hypertextdokument.
    \item Hypertext Transfer Protocol Secure (HTTPS): En säker version av HTTP
    som använder kryptering för att skydda dataöverföring.
    \item HTTP Basic Authentication: En enkel autentiseringsmekanism inom
    HTTP-protokollet.
    \item Structured Query Language (SQL): Ett språk som används för att
    hantera och manipulera relationsdatabaser.
    \item Hypertext Markup Language (HTML): Standardspråket för att skapa och
    strukturera sektioner på en webbsida.
    \item eXtensible Markup Language (XML): Ett flexibelt textformat för att
    organisera och transportera data
    \item JavaScript Object Notation (JSON): Ett lättviktigt datautbytesformat
    som är lätt för människor att läsa och skriva.
    \item Cascading Style Sheets (CSS): Stilspråket som används för att
    presentera ett dokument skrivet i HTML eller XML.
    \item Application Programming Interface (API): En uppsättning definitioner
    och protokoll för att bygga och integrera applikationsprogramvara.
    \item Representational State Transfer (REST eller RESTful): En
    arkitektonisk stil för att designa nätverksbaserade applikationer.
    \item RESTful API: Ett API som använder HTTP-protokollet och REST-principer
    för att skapa, läsa, uppdatera och ta bort data över webben. RESTful APIs
    är designade för att vara lättanvända och flexibla, vilket gör dem populära
    för webbapplikationsutveckling.
    \item Artificiell Intelligens (AI): Teknik som syftar till att skapa
    maskiner som kan utföra uppgifter som vanligtvis kräver mänsklig
    intelligens.
    \item OpenAI Whisper: En AI-baserad transkriberingsteknik från
    OpenAI.
\end{itemize}

\section{Syfte}
Syftet med YouTube Transcription Service är att erbjuda en automatiserad och
användarvänlig tjänst som genererar texttranskriptioner av YouTube-videor till
en åtkomlighet via samma webbsida. 

\section{Metod}

Plattformen bygger på en robust arkitektur där Apache Kafka hanterar köer av
transkriptionsförfrågningar medan Apache ZooKeeper koordinerar
konfigurationsdata. Go och Python används för att utveckla systemets
bakgrundstjänster där OpenAI:s Whisper står för själva
transkriberingsprocessen. Slutligen används MariaDB för att säkert lagra
transkriptioner och användardata medan Docker underlättar driftsättningen av
hela tjänsten på olika plattformar.

YouTube Transcription Service önskar göra videomaterial mer tillgängligt och
användbart. Genom att utnyttja skärningspunkten mellan avancerad AI och
användarcentrerad design, bidrar projektet till att demokratisera tillgången
till information.

\subsection{Systemöversikt}
Projektet syftar till att göra videomaterial mer tillgängligt genom att
kombinera avancerad AI-teknologi med användarcentrerad design. Plattformen
utnyttjar en robust arkitektur för att hantera transkriptionsprocessen från
början till slut.

\subsection{Klient-lager}

\subsubsection{Grafiskt gränssnitt}
Användargränssnittet är utformat för att vara intuitivt och lättanvänt,
möjliggörande för användare att enkelt ladda upp eller ange URL:er till
YouTube-videor för transkribering. Gränssnittet är utvecklat med HTML, CSS och
JavaScript för att säkerställa responsivitet och tillgänglighet över olika
enheter och webbläsare.

\subsubsection{Programmatiskt gränssnitt}
Det programmatiska gränssnittet förenklar användningen av systemet för
automatiserade system som ämnar utöka funktionaliteten. 

\subsection{Server-lager}

\subsubsection{Applikationsserver}
Bakgrundstjänsterna, utvecklade med Go och Python, hanterar logik för att ta
emot och behandla transkriptionsförfrågningar. Systemet integrerar Apache
Kafka för att hantera köer av transkriptionsförfrågningar, medan Apache
ZooKeeper koordinerar konfigurationsdata. Detta säkerställer effektiv
distribution och skalbarhet av tjänsten.

\subsubsection{Transkriberingsmotor}
OpenAI:s Whisper används för att utföra själva transkriberingen av videor.
Denna AI-drivna motor garanterar hög noggrannhet i transkriptionen, vilket är
avgörande för tillgängligheten och användbarheten av det transkriberade
innehållet.

\subsubsection{Databaslager}
MySQL används för att säkert lagra användardata och de genererade
transkriptionerna. Databasen är optimerad för att snabbt kunna hämta och lagra
data, vilket underlättar en smidig användarupplevelse.

\subsubsection{Säkerhet och autentisering}
Systemet implementeras med grundläggande HTTP-autentisering för att säkerställa
att endast auktoriserade användare skickar transkriptionsförfrågningar och
får tillgång till sina resultat. Vidare används HTTPS för att kryptera all
kommunikation mellan klienten och servern, skyddande användardata och
transkriptioner från obehörig åtkomst.

\subsection{Driftsättning}
Docker underlättar driftsättningen av hela tjänsten på olika
plattformar, vilket möjliggör en enhetlig och isolerad miljö för
applikationen. Detta bidrar till enklare underhåll, skalbarhet och
portabilitet av systemet.

\section{Kravspecifikation}

\subsection{Användningsfall}

\subsubsection{Skapa konto}
Jag som användare vill kunna skapa ett konto för att kunna begära
transkriberingar till YouTube-videor.

\subsubsection{Logga in}
Jag som användare vill kunna logga in för att kunna begära transkriberingar
till YouTube-videor.

\subsubsection{Skapa transkribering}
Jag som användare vill kunna skapa en transkribering genom att mata in en URL
till en YouTube-video för att få innehållet i videon utan att behöva kolla på
den.

\subsubsection{Hämta transkribering}
Jag som användare vill kunna hämta en transkribering genom att mata in en URL
till en YouTube-video för att få innehållet i videon utan att behöva kolla på
den.

\subsection{Tidsplan}
Se Figur \ref{fig:gantt}.
