\chapter{Introduktion}
\label{ch:introduction}

\section{Bakgrund}
\label{sec:background}
Med mängden användare av digitalt videomaterial på plattformar som YouTube och
Twitch, har behovet av tillgängliga textversioner blivit alltmer angeläget.
Denna utveckling reflekteras i akademiska studier som betonar vikten av
transkribering inom forskning, där det möjliggör en mer djupgående analys av
material och en förbättrad tillgänglighet för olika användargrupper
\cite{RevBlog2019}. Transkriberingens roll inom akademisk forskning
framhålls för att tillhandahålla en detaljerad och exakt representation av
muntliga data, vilket är avgörande för kvalitativ forskning
\cite{OxfordAcademicND}.

Trots tillgängligheten av olika transkriberingstjänster stöter många användare
på utmaningar relaterade till kostnad, noggrannhet och tillgänglighet.
Existerande lösningar kan kräva betydande tid för granskning och kan vara
budgetkrävande, vilket lägger hinder för forskare som arbetar under tidspress
\cite{RevBlog2019}. Projektet syftar till att tackla dessa utmaningar genom
att utveckla en kostnadseffektiv, noggrann och lättillgänglig
transkriberingstjänst för YouTube-videor. Effektiva transkriberingstjänster
kan spela en kritisk roll i att förbättra tillgången på forskningsdata och
stödja forskare i deras arbete, vilket ytterligare bekräftas av Happy Scribe
som framhäver transkriberingens värde i specialutbildningsforskning och hur
det kan förbättra förståelsen för komplexa termer och terminologi
\cite{HappyScribeND}. En transkriberingstjänst för YouTube-videor kan också
vara till nytta för personer med hörselnedsättning, vilket kan förbättra deras
tillgång till digitalt videomaterial. Sist så kan gemene man ha nytta av
transkriberingstjänster för att senare använda sammanfattningstjänsster för
att få en snabb överblick av materialet.

\section{Syfte}
\label{sec:purpose}
Syftet med projektet är att utveckla en tjänst som erbjuder en automatiserad
transkribering av YouTube-videor till text. 

\subsection{Nomenklatur}
\label{sec:nomenclature}
För att underlätta förståelsen av projektet och dess tekniska aspekter
presenteras här en kort förklaring av de termer som används i rapporten:
\begin{itemize}
    \item \textbf{Hypertext Transfer Protocol (HTTP)}: Ett protokoll för
    överföring av hypertextdokument \cite{RFC7231}.
    \item \textbf{Hypertext Transfer Protocol Secure (HTTPS)}: En säker
    version av HTTP som använder kryptering för att skydda dataöverföring
    \cite{RFC2818}.
    \item \textbf{HTTP Basic Authentication}: En enkel autentiseringsmekanism
    inom HTTP-protokollet \cite{RFC7617}.
    \item \textbf{Structured Query Language (SQL)}: Ett språk som används för
    att hantera och manipulera relationsdatabaser.
    \item \textbf{Hypertext Markup Language (HTML)}: Standardspråket för att
    skapa och strukturera sektioner på en webbsida.
    \item \textbf{eXtensible Markup Language (XML)}: Ett flexibelt textformat
    för att organisera och transportera data
    \item \textbf{JavaScript Object Notation (JSON)}: Ett lättviktigt
    datautbytesformat som är lätt för människor att läsa och skriva.
    \item \textbf{Cascading Style Sheets (CSS)}: Stilspråket som används för
    att presentera ett dokument skrivet i HTML eller XML.
    \item \textbf{Application Programming Interface (API)}: En uppsättning
    definitioner och protokoll för att bygga och integrera
    applikationsprogramvara.
    \item \textbf{Representational State Transfer (REST eller RESTful)}: En
    arkitektonisk stil för att designa nätverksbaserade applikationer.
    \item \textbf{RESTful API}: Ett API som använder HTTP-protokollet och
    REST-principer för att skapa, läsa, uppdatera och ta bort data över
    webben. RESTful APIs är designade för att vara lättanvända och flexibla,
    vilket gör dem populära för webbapplikationsutveckling.
    \item \textbf{Artificiell Intelligens (AI)}: Teknik som syftar till att
    skapa maskiner som kan utföra uppgifter som vanligtvis kräver mänsklig
    intelligens.
    \item \textbf{OpenAI Whisper}: En AI-baserad transkriberingsteknik från
    OpenAI.
    \item \textbf{Apache Kafka}: En distribuerad strömhanteringsplattform som
    används för att bygga realtidsdataflöden och realtidsapplikationer.
    \item \textbf{Kafka Topic}: En kategori eller flöde av meddelanden i
    Kafka.
\end{itemize}

\section{Metod och kravställning}
\label{sec:method-and-requirements}

\subsection{Kravställning}
\label{sec:requirements}
Projektet har identifierat följande krav för att uppnå syftet:
\begin{itemize}
    \item \textbf{Tillgänglighet}: Plattformen ska vara tillgänglig för alla
    användare oavsett teknisk kompetens genom att erbjuda ett webbaserat
    grafiskt gränssnitt. Systemet ska även erbjuda ett programmatiskt
    gränssnitt för automatiserade system.
    \item \textbf{Skalbarhet}: Plattformen ska vara skalbar för att hantera
    det krav som ställs då användning av AI-verktyg kan vara resurskrävande
    \cite{GreenSoftwareFoundation2024CanAITrulyBeGreen}.
    \item \textbf{Noggrannhet}: Plattformen ska erbjuda en inställning för att
    välja noggrannhetsnivå för transkribering för bekostnad av tid.
    \item \textbf{Persistering}: Plattformen ska spara transkriberingar för
    att möjliggöra åtkomst och analys av innehållet.
    \item \textbf{Säkerhet}: Plattformen ska skydda kommunikationen och
    transkriberingar från obehörig åtkomst genom att använda HTTPS för att
    kryptera all kommunikation mellan klienten och servern. Grundläggande
    HTTP-autentisering ska användas för att säkerställa att endast
    auktoriserade användare skickar transkriptionsförfrågningar och får
    tillgång till texterna.
\end{itemize}

\subsection{Användningsfall}
\label{sec:use-cases}

\begin{itemize}
    \item \textbf{Logga in:} Jag som användare vill kunna logga in för att
    kunna logga in för att kunna begära transkriberingar till YouTube-videor.
    \item \textbf{Skapa transkribering:} Jag som användare vill kunna skapa en
    transkribering genom att mata in en URL till en YouTube-video för att få
    innehållet i videon utan att behöva kolla på den.
    \item \textbf{Hämta transkribering:} Jag som användare vill kunna hämta en
    transkribering genom att mata in en URL till en YouTube-video för att få
    innehållet i videon utan att behöva kolla på den.
\end{itemize}

\section{Avgränsningar}
\label{sec:limitations}
Projektet fokuserar på att utveckla en transkriberingstjänst för
YouTube-videor. Detta innebär att tjänsten inte kommer att stödja andra
plattformar som Twitch eller Vimeo. Projektet kommer inte heller att stödja
transkribering av ljudfiler eller andra typer av videofiler. Inget krav på
språkstöd kommer att implementeras i projektet, vilket innebär att tjänsten
inte behöver stödja transkribering av videor på andra språk än engelska.

Tjänsten behöver inte stödja att användare kan registrera sig och skapa
användarkonton. Då tjänsten har autentisering som krav så kommer istället
hårdkodade användarnamn och lösenord att användas för att autentisera
klienter.

\subsection{Metod}
\label{sec:method}
Projektet nyttjar GitHub för versionshantering och projektledning. Issues och
pull requests används för att hantera arbetsuppgifter och kodgranskning. 
GitHubs inbyggda projektverktyg används för att hantera projektets tidsplan 
som beskrivs vidare i \ref{sec:timeplan}. 

\subsection{Tidsplan}
\label{sec:timeplan}
Se Figur \ref{fig:gantt}.
