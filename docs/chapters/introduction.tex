\chapter{Introduktion}

\section{Bakgrund}
Med mängden användare av digitalt videomaterial på plattformar som YouTube har behovet av tillgängliga textversioner blivit alltmer angeläget. Denna utveckling reflekteras i akademiska studier som betonar vikten av transkribering inom forskning, där det möjliggör en mer djupgående analys av material och en förbättrad tillgänglighet för olika användargrupper (Rev Blog, 2019). Transkriberingens roll inom akademisk forskning har också framhållits för att tillhandahålla en detaljerad och exakt representation av muntliga data, vilket är avgörande för kvalitativ forskning (Oxford Academic, n.d.).

Trots tillgängligheten av olika transkriberingstjänster stöter många användare på utmaningar relaterade till kostnad, noggrannhet och tillgänglighet. Existerande lösningar kan kräva betydande tid för granskning och kan varabudgetkrävande, vilket lägger hinder förforskare som arbetar under tidspress (Rev Blog, 2019). Vårt projekt syftar till att tackla dessa utmaningar genom att utveckla en kostnadseffektiv, noggrann och lättillgänglig transkriberingstjänst för YouTube-videor. Effektiva transkriberingstjänster kan spela en kritisk roll i att förbättra tillgången på forskningsdata och stödja forskare i deras arbete, vilket ytterligare bekräftas av Happy Scribe som framhäver transkriberingens värde i speciaalutbildningsforskning och hur det kan förbättra förståelsen för komplexa termer och terminologi (Happy Scribe, n.d.).

\subsection{Nomenklatur}
JavaScript Object Notation (vidare känt som JSON).

\section{Syfte}
Syftet med YouTube Transcription Service är att erbjuda en automatiserad och användarvänlig tjänst som genererar texttranskriptioner av YouTube-videor till en textfil skickad till användarens email. Detta möjliggör ökad tillgänglighet och breddar användningsområdena för videoinnehåll.

\section{Metod}
Plattformen bygger på en robust arkitektur där Apache Kafka hanterar köer av transkriptionsförfrågningar medan Apache ZooKeeper koordinerar konfigurationsdata. Go och Python används för att utveckla systemets bakgrundstjänster där OpenAI:s Whisper står för själva transkriberingsprocessen. Slutligen används MariaDB för att säkert lagra transkriptioner och användardata medan Docker underlättar driftsättningen av hela tjänsten på olika plattformar.

YouTube Transcription Service önskar göra videomaterial mer tillgängligt och användbart. Genom att utnyttja skärningspunkten mellan avancerad AI och användarcentrerad design, bidrar projektet till att demokratisera tillgången till information.
