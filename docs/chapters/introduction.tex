\chapter{Introduktion}

\section{Bakgrund}
Med mängden användare av digitalt videomaterial på plattformar som YouTube har behovet av tillgängliga textversioner blivit alltmer angeläget. Denna utveckling reflekteras i akademiska studier som betonar vikten av transkribering inom forskning, där det möjliggör en mer djupgående analys av material och en förbättrad tillgänglighet för olika användargrupper (Rev Blog, 2019). Transkriberingens roll inom akademisk forskning har också framhållits för att tillhandahålla en detaljerad och exakt representation av muntliga data, vilket är avgörande för kvalitativ forskning (Oxford Academic, n.d.).

Trots tillgängligheten av olika transkriberingstjänster stöter många användare på utmaningar relaterade till kostnad, noggrannhet och tillgänglighet. Existerande lösningar kan kräva betydande tid för granskning och kan varabudgetkrävande, vilket lägger hinder förforskare som arbetar under tidspress (Rev Blog, 2019). Vårt projekt syftar till att tackla dessa utmaningar genom att utveckla en kostnadseffektiv, noggrann och lättillgänglig transkriberingstjänst för YouTube-videor. Effektiva transkriberingstjänster kan spela en kritisk roll i att förbättra tillgången på forskningsdata och stödja forskare i deras arbete, vilket ytterligare bekräftas av Happy Scribe som framhäver transkriberingens värde i speciaalutbildningsforskning och hur det kan förbättra förståelsen för komplexa termer och terminologi (Happy Scribe, n.d.).

\subsection{Nomenklatur}
AI (Artificiell Intelligens): Teknik som syftar till att skapa maskiner som kan utföra uppgifter som vanligtvis kräver mänsklig intelligens.

API (Application Programming Interface): En uppsättning definitioner och protokoll för att bygga och integrera applikationsprogramvara.

@AI Whisper: En AI-baserad transkriberingsteknik från OpenAI.

Backend: Server-sidan av en webbapplikation, där affärslogik, databasinteraktioner och serverns programvara hanteras.

Basic HTTP Authentication: En enkel autentiseringsmekanism inom HTTP-protokollet.

CSS (Cascading Style Sheets): Stilspråket som används för att presentera ett dokument skrivet i HTML eller XML.

Databas: Ett systematiskt organiserat elektroniskt register som effektivt kan söka, välja och lagra data.

Domän: Identifiering av en specifik enhet av administrativ autonomi, auktoritet eller kontroll inom domännamnssystemet på Internet.

Frontend: Den del av en webbapplikation som användaren interagerar med direkt.

Full-stack Web Application: En applikation som inkluderar klient, server och databaslager.

Git: Ett verktyg för versionshantering.

GitHub: En plattform för hosting och samarbete kring programvaruutvecklingsprojekt.

Go (GoLang): Ett programmeringsspråk designat för att vara enkelt, effektivt och skalbart.

HTML (Hypertext Markup Language): Standardspråket för att skapa och strukturera sektioner på en webbsida.

HTTP (Hypertext Transfer Protocol): Ett protokoll för överföring av hypertextdokument.

HTTPS (Hypertext Transfer Protocol Secure): En säker version av HTTP som använder kryptering för att skydda dataöverföring.

JS (JavaScript): Ett programmeringsspråk som tillåter interaktivitet på webbsidor.

JSON (JavaScript Object Notation): Ett lättviktigt datautbytesformat som är lätt för människor att läsa och skriva.

Klient: Programvara eller enhet som skickar förfrågningar till en server för att hämta eller använda resurser.

Linux: Ett fritt och öppet källkodsoperativsystem baserat på Unix.

macOS: Apples operativsystem för Mac-datorer.

MariaDB: En databasförvaltare som är en förgrening av MySQL.

MySQL: Ett relationsdatabassystem som använder SQL för databasadministration.

Öppen källkod: Programvara vars källkod är tillgänglig för allmänheten att använda, modifiera och distribuera.

Privat server: En serverresurs som exklusivt används av en organisation eller individ.

Python: Ett högnivåprogrammeringsspråk känt för sin läsbarhet och effektivitet.

Relationsmodellen: En databasmodell som definierar en databas som en samling av relationer, eller tabeller.

Server: En dator eller programvara som tillhandahåller dataresurser eller tjänster till klienter över ett nätverk.

SQL (Structured Query Language): Ett språk som används för att hantera och manipulera relationsdatabaser.

Three-Tier Architecture: En programvaruarkitektur som delar upp ett system i tre lager.

URL (Uniform Resource Locator): En adress som anger platsen för en resurs på Internet.

**VSCode (Visual Studio Code): En fri och öppen källkodsredigerare utvecklad av Microsoft för Windows, Linux och macOS.

Webbläsare: Ett program som används för att hämta, presentera och navigera webbresurser på WWW.

Webbsida: Ett dokument tillgängligt via en webbläsare på World Wide Web, ofta skrivet i HTML och tillgängligt via HTTP eller HTTPS.

Windows: Ett operativsystem från Microsoft för persondatorer och servrar.

WWW (World Wide Web): Ett globalt informationssystem på Internet där dokument och andra webbresurser är interlänkade via URL:er och kan nås via webbläsare.

XML (Extensible Markup Language): Ett flexibelt textformat för att organisera och transportera data

\section{Syfte}
Syftet med YouTube Transcription Service är att erbjuda en automatiserad och användarvänlig tjänst som genererar texttranskriptioner av YouTube-videor till en textfil skickad till användarens email. Detta möjliggör ökad tillgänglighet och breddar användningsområdena för videoinnehåll.

\section{Metod}
Plattformen bygger på en robust arkitektur där Apache Kafka hanterar köer av transkriptionsförfrågningar medan Apache ZooKeeper koordinerar konfigurationsdata. Go och Python används för att utveckla systemets bakgrundstjänster där OpenAI:s Whisper står för själva transkriberingsprocessen. Slutligen används MariaDB för att säkert lagra transkriptioner och användardata medan Docker underlättar driftsättningen av hela tjänsten på olika plattformar.

YouTube Transcription Service önskar göra videomaterial mer tillgängligt och användbart. Genom att utnyttja skärningspunkten mellan avancerad AI och användarcentrerad design, bidrar projektet till att demokratisera tillgången till information.
