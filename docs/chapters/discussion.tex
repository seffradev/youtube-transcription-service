\chapter{Diskussion}

Reflektera över de tekniska framstegen inom AI och transkriberingsteknik, är det viktigt att erkänna begränsningarna av dessa system när det kommer till tolkningen av kroppsspråk. Trots att AI-baserade transkriberingsverktyg som OpenAI:s Whisper kan uppnå hög noggrannhet i att identifiera och omvandla talat ord till text, finns det en dimension av mänsklig kommunikation som fortfarande utmanar dagens teknologi: kroppsspråket.

Kroppsspråket, som inkluderar gester, ansiktsuttryck, och tonfall, spelar en kritisk roll i hur budskap uppfattas och förstås av mottagaren. Denna icke-verbala kommunikation kan bära en mängd information som kompletterar, förstärker eller till och med motsäger det som sägs verbalt. Detta innebär att även den mest noggranna texttranskriptionen kan missa nyanser och betoningar som förmedlas genom kroppsspråk.

Diskussion om Tekniska och Mänskliga Utmaningar
Denna begränsning leder till en viktig diskussion om balansen mellan tekniska lösningar och mänsklig tolkning. Även om teknologin fortsätter att utvecklas och blir allt mer sofistikerad, finns det en värdefull aspekt av mänsklig förståelse och tolkning som inte lätt kan replikeras av maskiner. Det ställer frågor om hur vi bäst kan använda teknologin som ett verktyg för att komplettera mänsklig förmåga snarare än att ersätta den.

\section{Vidare arbeten}
Utöver att utforska möjligheter för att förbättra teknikens förmåga att tolka kroppsspråk, finns det andra potentiella områden för vidareutveckling av systemet:

Segmenterad Transkribering: Ett förslag är att möjliggöra för användare att välja specifika delar av en video för transkribering genom att ange ett tidsintervall. Detta skulle inte bara göra systemet mer flexibelt utan också mer kostnadseffektivt och tidsbesparande för användaren.

Flera Format för Transkriptioner: Att erbjuda transkriptioner i olika format som JSON, XML, Markdown, LaTeX, och punktform skulle kunna tillgodose de varierande behoven hos olika användargrupper. Forskare kan föredra LaTeX för akademiska publikationer, medan utvecklare kan föredra JSON eller XML för datamanipulation.

Dessa tillägg skulle inte bara utöka systemets användbarhet utan också dess tillgänglighet, vilket stärker dess roll i att demokratisera informationstillgång.

Tekniska och Mänskliga Utmaningar
Att integrera kroppsspråk i transkriptionen representerar en betydande teknisk utmaning men öppnar också upp för diskussioner om framtida innovationer inom AI och maskininlärning. Utvecklingen av algoritmer som kan identifiera och tolka kroppsspråk skulle kunna revolutionera hur vi interagerar med AI-system och öka deras förmåga att förstå mänskliga nyanser i kommunikation.

Sociala och Etiska Överväganden
När vi utforskar dessa nya territorier, måste vi också vara medvetna om de sociala och etiska implikationerna, särskilt när det gäller integritet och kulturella skillnader i kroppsspråk. Att säkerställa att dessa teknologier utvecklas på ett sätt som respekterar användarnas integritet och kulturella mångfald blir avgörande.