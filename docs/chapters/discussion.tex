\chapter{Diskussion}

\section{Analys}
För att säkerställa att transkriberingar sker i rätt ordning, så läggs en
transkriberingsförfrågan i en kö och läggs upp i databasen med status
''pending''. Den tas sedan upp av en transkriberingstjänst och när den är
klar, så uppdateras statusen till ''completed''. Om en transkribering är
markerad med en status som inte är ''pending'', exempelvis ''completed'' eller
''in progress'', så kommer inte en annan transkriberingstjänst att processa
den. Om en transkribering är markerad med status ''failed'' kan den återupptas
om den efterfrågas igen via webbapplikationen.

Utan hanteringen på detta sätt skulle det kunna uppstå problem med att flera
transkriberingstjänster försöker transkribera samma video samtidigt. Det
skulle kunna leda till att samma video transkriberas flera gånger, vilket
kostar resurser. Statusarna skyddar även mot att en transkribering görs om
då en transkriberingsnod kraschar och går över listan över
transkriberingsförfrågningar.

Klienterna får en sammanställd bild via den gemensamma databasen som sköts av
transkriberingstjänsten och webbapplikationen. Där lagras alla
transkriberingar och klienter kan se alla transkriberingar som är gjorda av
andra klienter. Noderna i systemet är oberoende av varandra
och kan köras på olika maskiner, med en stark särkoppling från allt utom
databas och meddelandekön. Det gör att den verklighetsförankring noderna
kräver ställs på MySQL och Apache Kafka.

Då systemet hanterar meddelanden och händelser är det lätt att misstänka att
det handlar om eventuell konsekvens i systemet, men det är inte fallet. 
Eftersom klienterna inte får olika svar beroende på vilken nod som svarar, så
är det inte en inkonsekvens i systemet. När en transkibering är klar så
uppdateras statusen i databasen och alla klienter får därav den nya statusen
lika snabbt som om den hade fått svaret direkt från transkriberingstjänsten.

Om en nod kraschar så kommer den att startas om av Docker Compose. Detta
sker automatiskt och är en del av konfigurationen i Docker Compose. Beroende
på vilken nod som kraschar så har det olika stora inverkningar på systemet.
Om webbapplikationen kraschar så kommer klienter inte kunna använda förfråga
transkriberingar. Om transkriberingstjänsten kraschar så kommer inga 
transkriberingar att bli gjorda. Om databasen kraschar så kommer inga
transkriberingar att kunna lagras eller hämtas. Om Apache Kafka kraschar så
kommer inga meddelanden att kunna skickas eller tas emot. För allmändriften
av systemet så är det viktigt att Apache Kafka och databasen är tillgängliga,
medan webbapplikationen och transkriberingstjänsten startar om sig själva. 
Webbapplikationen kan fortsätta skicka meddelanden till Apache Kafka så 
transkriberingstjänsten kan fortsätta att transkribera när den startat om.

\section{Projektarbetet}
Projektet valdes att skala upp i svårighetsgrad utöver en trelagersarkitektur
och en enkel webbapplikation. Detta gjordes för att utmana oss själva och
få en djupare förståelse för hur många moderna system kan se ut. Detta
innebar att vi behövde lära oss att använda Apache Kafka, Apache Zookeeper,
Docker Compose och Confluents \cite{Confluent2024} bibliotek i Go och Python
för kommunikation med Apache Kafka. Utöver det valde vi att använda ett
programmeringsspråk som vi inte hade använt tidigare, nämligen Go. Det ledde
till mycket läsning och förberedelse innan projektet tog fart. 

Utöver språk och teknologivalen så använde vi GitHub flitigt för att
hantera kod och dokumentation. Vi använde även GitHub Actions för att
automatisera generation av denna rapport, författad i \LaTeX. Pull requests
och code reviews användes för att säkerställa kvaliteten på koden och issues
användes för att hantera uppgifter, buggar och ansvarsområden.

I mån av tid har Figur \ref{fig:transcription-flowchart} och Figur 
\ref{fig:system-flowchart} inte fått med de potentiella fel som kan uppstå
i systemet. Likaså kvarstår \ref{fig:gantt} i mörkt tema men går att zooma in
på för att se detaljer.

\section{Vidare arbeten}
Utöver att utforska möjligheter för att förbättra teknikens förmåga att tolka
kroppsspråk, finns det andra potentiella områden för vidareutveckling av
systemet:

\subsection{Caching}
För att förbättra prestanda och användarupplevelse, skulle det vara möjligt
att implementera en cache för att lagra transkriptioner som redan har gjorts.
Detta skulle minska belastningen på systemet och förbättra svarstiden för
användare. Enbart om användarantalet är stort och man ser ett mönster i
efterfrågade transkriptioner skulle detta vara en relevant förbättring. 
Exempelvis med en Redis-databas. 

\subsection{Sammanfattning och analys}
Att integrera en sammanfattning och analys av transkriptioner skulle kunna
öka systemets användbarhet och värde för användare. Detta skulle kunna
innefatta en sammanfattning av de viktigaste punkterna i transkriptionen, en
analys av tonfall och känslomässiga nyanser. Detta skulle kunna vara särskilt
användbart för forskare och journalister som behöver snabbt få en överblick av
en videos innehåll.

\subsection{Segmenterad transkribering}
Ett förslag är att möjliggöra för användare att
välja specifika delar av en video för transkribering genom att ange ett
tidsintervall. Detta skulle inte bara göra systemet mer flexibelt utan också
mer kostnadseffektivt och tidsbesparande för användaren.

\subsection{Flera format för transkriptioner}
Att erbjuda transkriptioner i olika format
som JSON, XML, Markdown, \LaTeX, och punktform skulle kunna tillgodose de
varierande behoven hos olika användargrupper. Forskare kan föredra \LaTeX för
akademiska publikationer, medan utvecklare kan föredra JSON eller XML för
datamanipulation. Att inkludera funktionen för översättning till andra språk skulle ytterligare utöka systemets användbarhet och tillgänglighet. Detta skulle möjliggöra för en ännu bredare användargrupp att dra nytta av tjänsten vilket ytterligare stärker systemets roll i att demokratisera tillgången till information.

Dessa tillägg skulle inte bara utöka systemets användbarhet utan också dess
tillgänglighet, vilket stärker dess roll i att demokratisera
informationstillgång.

\subsection{Autentisering och säkerhet}
I dagsläget är systemet öppet för alla att använda utan möjlighet till
att autentisera ett privat konto. Detta innebär att användare inte kan spara
egna transkriptioner och att alla transkriptioner är publika. Detta skulle
kunna vara en viktig del av systemet för att säkerställa att användare kan
spara och komma åt sina transkriptioner. Detta skulle också möjliggöra för
användare att spara sina inställningar och preferenser för framtida användning.
Vidare skulle det också möjliggöra för användare att dela transkriptioner med
andra användare, vilket skulle kunna vara användbart för forskning och
samarbete. Ett sådant tillägg skulle ge användare ökad kontroll över och
integritet i sina transkriptioner.

Ytterligare säkerhetsåtgärder skulle också kunna övervägas för att skydda
transkriptioner och användarinformation. Detta skulle kunna innefatta
kryptering av transkriptioner, en tydligare integritetspolicy och möjligheten
för användare att radera sina transkriptioner från systemet.

\subsection{Monetärisering}
Då systemet nyttjar AI-modeller och andra resurser, skulle det kunna vara
möjligt att överväga monetärisering av systemet för att täcka kostnader för
drift och underhåll. Detta skulle kunna innefatta en prenumerationsmodell för
användare som vill ha tillgång till avancerade funktioner, eller en modell
där användare betalar per transkription. Detta skulle kunna vara en viktig del
av att säkerställa systemets långsiktiga hållbarhet och tillgänglighet.

Ett alternativ är att användare kan köpa polletter som de kan använda för att
transkribera sina videor. Detta skulle kunna vara en mer flexibel modell som
tillåter användare att köpa polletter i olika storlekar beroende på deras
behov. Detta skulle också kunna vara en mer rättvis modell för användare som
endast behöver transkribera enstaka videor. Om användare nyttjar en
transkription av en video begärd av en annan användare, skulle polletter
överföras från den begärande användaren till den transkriberande användaren.
Det uppmuntrar användare att transkribera videor för andra användare och
skapar en mer rättvis modell för användare som transkriberar fler videor än de
själva begär.

\section{Tekniska och mänskliga utmaningar}
I en reflektion över de tekniska framstegen inom AI och transkriberingsteknik,
är det viktigt att erkänna begränsningarna av dessa system när det kommer till
tolkningen av kroppsspråk. Trots att AI-baserade transkriberingsverktyg som
OpenAI:s Whisper kan uppnå hög noggrannhet i att identifiera och omvandla
talat ord till text \cite{OpenAIWhisper}, finns det en dimension av mänsklig
kommunikation som fortfarande utmanar dagens teknologi: kroppsspråket.

Kroppsspråket, som inkluderar gester, ansiktsuttryck, och tonfall, spelar en
kritisk roll i hur budskap uppfattas och förstås av mottagaren. Denna
icke-verbala kommunikation kan bära en mängd information som kompletterar,
förstärker eller till och med motsäger det som sägs verbalt. Detta innebär att
även den mest noggranna texttranskriptionen kan missa nyanser och betoningar
som förmedlas genom kroppsspråk.

Denna begränsning leder till en viktig diskussion om balansen mellan tekniska
lösningar och mänsklig tolkning. Även om teknologin fortsätter att utvecklas
och blir allt mer sofistikerad, finns det en värdefull aspekt av mänsklig
förståelse och tolkning som inte lätt kan replikeras av maskiner. Det ställer
frågor om hur vi bäst kan använda teknologin som ett verktyg för att
komplettera mänsklig förmåga snarare än att ersätta den.

Att integrera kroppsspråk i transkriptionen representerar en betydande teknisk
utmaning men öppnar också upp för diskussioner om framtida innovationer inom
AI och maskininlärning. Utvecklingen av algoritmer som kan identifiera och
tolka kroppsspråk skulle kunna revolutionera hur vi interagerar med AI-system
och öka deras förmåga att förstå mänskliga nyanser i kommunikation.

När dessa nya territorier utforskas, måste också medvetenhet kring de sociala
och etiska implikationerna upprätthållas, särskilt vad gäller integritet och
kulturella skillnader i kroppsspråk. Att dessa teknologier utvecklas på ett
sätt som respekterar användarnas integritet och kulturella mångfald blir
avgörande.
